\documentclass{article}
\usepackage[utf8]{inputenc}

\title{Ejercisios Proc}
\author{Francisco Javier Atondo Nubes }
\date{6/10/2022}

\begin{document}

\maketitle

\begin{tab}

\section{3.19}


In many languages, procedures must be created and named at the
same time. Modify the language of this section to have this property by replacing the
proc expressionwith a letproc expression.\\


    Antes \= \> \> \> \> \> \> \> \> \> \> \> \> \> Despues\\
    Let f = proc (x) – (x,4) \= \> \> letproc f(x) – (x,4)\\
    In (f (f 5)) \= \> \> \> \> \> \> \> \> \> \> \> \> in (f (f 5))\\




Sintaxis Concreta\\
Quitamos produccion:\\
Expression := proc (Identifier) Expression\\ \\
Agregamos:\\
Expression := leftproc Identifier (Identifier) Expression in Expression\\
\= \>	Letproc-exp (fun param body exp1)\\

Semantica\\
ExpVal = Int + Bool\\
DenVal = Int + Bool + Proc\\
(value-of (letproc-exp name param body exp1 ) env)\\
= (value-of exp1 ( [name = (proc-val (procedure param body env) ] ) env )\\
\newpage
\section{3.20}

In PROC, procedures have only one argument, but one can get the effect of multiple argument procedures by using procedures that return other procedures. For example, one might write code like\\ \\
let f = proc (x) proc (y) ...\\
in ((f 3) 4)\\ \\

This trick is called Currying, and the procedure is said to be Curried. Write a Curried procedure that takes two arguments and returns their sum. You can write x + y in our language by writing −(x,−(0, y)).\\ \\

Sintaxis Concreta:\\
Definimos como:\\
Expression ::= proc - ( Expression x proc -( Expression y ) )\\

Semantica:\\
ExpVal = Int + Proc\\
(value-of exp1( proc-val ( param body ) env ) )

\newpage
\section{ 3.21 } 
Extend the language of this section to include procedures with multiple arguments and calls with multiple operands, as suggested by the grammar \\
Expression ::= proc ({Identifier}∗(,)) Expression\\
Expression ::= (Expression {Expression}∗)\\ \\

Sintaxis Concreta:\\
Expression ::= proc ( ( {Identifier} (Identifier)' ) ( Expression ( Expression ) )' )\\
Agregamos:\\
\= \> let n\\
\= \> in ( ( body )'(n) ( var )'(n) )\\ \\

Sintaxis Semantica:\\
( call-exp (  value-of (const-exp n ) (let-exp n exp body ) ) p )\\



\newpage
\section{3.25}
Exercise 3.25 [] The tricks of the previous exercises can be generalized to show that we can define any recursive procedure in PROC. Consider the following bit of code:\\
\= \> let makerec = proc (f)\\
\= \> \> let d = proc (x)\\
\= \> \>\> proc (z) ((f (x x)) z)\\
\= \> \>in proc (n) ((f (d d)) n)\\
\= \> in let maketimes4 = proc (f)\\
\= \> \> proc (x)\\
\= \> \> if zero?(x)\\
\= \> \> then 0\\
\= \> \> else -((f -(x,1)), -4)\\
\= \> in let times4 = (makerec maketimes4)\\
\= \> \> in (times4 3)\\
Show that it returns 12.

\newpage
\section{3.27}
Add a new kind of procedure called a traceproc to the language. A traceproc works exactly like a proc, except that it prints a trace message on entry and on exit.

let traceproc = proc print ( procedure p )\\
in let 
    
\end{tab}
\end{document}
